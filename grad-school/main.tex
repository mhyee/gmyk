\documentclass{beamer}
%\documentclass[12pt,handout]{beamer}

\usepackage[T1]{fontenc}
\usepackage{ae,aecompl}
\usepackage{pgfpages}

%\setbeameroption{show notes on second screen}
%\setbeameroption{show notes on second screen=bottom}

\usetheme{Rochester}

\setbeamertemplate{footline}
{%
  \hbox{%
    \begin{beamercolorbox}[wd=0.9\paperwidth,ht=2.25ex,dp=1ex,center]{section in head/foot}%
      \insertnavigation{0.9\paperwidth}
    \end{beamercolorbox}%
    \begin{beamercolorbox}[wd=0.1\paperwidth,ht=2.25ex,dp=1ex,right]{section in head/foot}%
      \insertframenumber{} / \insertpresentationendpage\hspace*{2ex}
    \end{beamercolorbox}%
  }%
}

\title{Graduate School}
\subtitle{What I Wish I Knew in 3A}
\author[Ming-Ho Yee]{Ming-Ho Yee\\ \texttt{mhyee.com}}
\institute{University of Waterloo\\ Software Engineering 2014}
\date{\today}

\begin{document}

\maketitle

\AtBeginSection[]
{%
  \begin{frame}
    \frametitle{Outline}
    \tableofcontents[currentsection]
  \end{frame}
}

\section{Introduction}
\begin{frame}
  \frametitle{What This Presentation Is About}
  \begin{itemize}
      \item What I wish I knew in 3A
      \item What I learned since 3A
      \item For computer science and software engineering students
        \begin{itemize}
          \item Somewhat applicable for math, science, and engineering
          \item Probably less helpful for other fields
        \end{itemize}
  \end{itemize}
\end{frame}

\begin{frame}
  \frametitle{My Experiences}
  \begin{itemize}
      \item 4B Software Engineering, class of 2014
      \item Research experiences
        \begin{itemize}
            \item Co-op, URA, FYDP
            \item Applied to graduate schools in Canada and USA
        \end{itemize}
      \item Industry experiences
        \begin{itemize}
          \item Telecommunications, web development startup
          \item Most recently: Visual C++ compiler at Microsoft
        \end{itemize}
  \end{itemize}
\end{frame}
\note[itemize]
{%
  \item Caveat: I'm not a graduate student
  \item But I have some research and some industry experience
}

\section{Explaining Graduate School}
\begin{frame}
  \frametitle{Course-based Degree}
  \begin{itemize}
    \item Pay tuition and take courses (consume knowledge)
    \item Examples: law school, med school, MBA, MMath, MEng
  \end{itemize}
\end{frame}

\begin{frame}
  \frametitle{Research-based Degree}
  \begin{itemize}
    \item Paid to do research (produce new knowledge)
    \item Examples: MMath, MASc, PhD
    \item
      \href{http://matt.might.net/articles/phd-school-in-pictures/}{%
        \color{blue}{Matt Might's illustrated guide to a Ph.D.}
      }
  \end{itemize}
  \vspace{2em}
  \pause After this point, ``graduate school'' refers to \alert{research}
\end{frame}
\note[itemize]
{%
  \item Matt Might is a professor at the University of Utah
  \item Note that at Waterloo, MMath is used for both course and thesis master's
        degrees
  \item In the US, ``master's'' typically means course-based
}

\section{Reasons for Graduate School}
\begin{frame}
  \frametitle{Reasons for Graduate School}
  \href{http://pgbovine.net/why-pursue-PhD.htm}{%
    \color{blue}{From Philip Guo: trade money for freedom}
  }
  \begin{itemize}
    \item Make a name for yourself
    \item Fail in a safe environment
    \item Choose from more jobs
  \end{itemize}
\end{frame}
\note[itemize]
{%
  \item There are many different reasons
  \item Many of them can be personal
  \item Philip Guo (professor at University of Rochester) offers three practical reasons
}

\begin{frame}
  \frametitle{Make a Name for Yourself}
  \begin{itemize}
    \item Formulate, execute, and sell your ideas
    \item Entire process from start to finish, on your own
  \end{itemize}
\end{frame}
\note[itemize]
{%
  \item Come up with an idea, implement or test it, then write papers and give
        presentations
  \item ``On your own,'' but you get mentoring from peers and professors
  \item Startups are another way to do this
}

\begin{frame}
  \frametitle{Fail in a Safe Environment}
  \begin{itemize}
    \item Failure is an opportunity for growth
    \item Failure can inspire grit, tenacity, and perseverance
    \item Expected to fail repeatedly, without hurting career
  \end{itemize}
\end{frame}
\note[itemize]
{%
  \item ``What doesn't kill you makes you stronger''
  \item These are good skills to have in any job
  \item In other jobs, failure may result in consequences (getting fired, not
        getting promoted, losing your startup, etc.)
  \item Though you don't necessarily need failure to develop these skills
}

\begin{frame}
  \frametitle{Choose From More Jobs}
  \begin{itemize}
    \item Co-op experiences makes us very employable at graduation
    \item A PhD can open even more opportunities
      \begin{itemize}
        \item Corporate research, academic research, teaching
      \end{itemize}
  \end{itemize}
\end{frame}
\note[itemize]
{%
  \item Many of us get full-time offers by our last co-op term
  \item New opportunities are also different
  \item Caveat: getting out of touch after you get your PhD
  \item Can maintain skills with internships
}

\begin{frame}
  \frametitle{Reasons Against Graduate School}
  \begin{itemize}
    \item Like working in industry
    \item Lower salary
    \item Might not increase starting salary in industry
  \end{itemize}
\end{frame}
\note[itemize]
{%
  \item Like working in industry or don't like research
  \item Salary can be 3 to 4 times lower
  \item Graduate degree might be useless, depending on position and company
  \item Even if it does increase starting salary, it's offset by the time
        investment
}

\section{Applying to Graduate School}
\begin{frame}
  \frametitle{Job Application for a Research Position}
  \begin{itemize}
    \item Apply 1 year before expected start
    \item Publications (research experience)
    \item Potential (to do research)
    \item Other qualities:
      \begin{itemize}
        \item \href{http://matt.might.net/articles/successful-phd-students/}{%
          \color{blue}{Perseverance, tenacity, cogency}
        }
      \end{itemize}
  \end{itemize}
\end{frame}
\note[itemize]
{%
  \item For SE students, this will be during your last co-op
  \item Can be difficult if you're not in Waterloo
  \item Perseverance and tenacity: Getting a PhD is open-ended, uncertain, and
        full of rejection/failure
  \item Cogency: Ability to articulate ideas, in papers and presentations
}

\begin{frame}
  \frametitle{Research Experience}
  \begin{itemize}
    \item Work with different professors
    \item Opportunities:
      \begin{itemize}
        \item Co-op, URA, FYDP, SE 499
      \end{itemize}
  \end{itemize}
\end{frame}
\note[itemize]
{%
  \item Working with different professors gives you multiple reference letters
  \item May have to go ``outside JobMine'' for research co-ops
  \item URA can be difficult, since it's on top of your course work
}

\begin{frame}
  \frametitle{Letters of Recommendation}
  \begin{itemize}
    \item Need 2-3 letters
    \item Strong, academic references
      \begin{itemize}
        \item Professor you've done research with
      \end{itemize}
    \item ``Course'' letter
      \begin{itemize}
        \item Professor whose course you excelled in
      \end{itemize}
    \item Industry letter
      \begin{itemize}
        \item Demonstrated qualities appropriate for research
        \item Some universities won't accept these letters
      \end{itemize}
  \end{itemize}
\end{frame}
\note[itemize]
{%
  \item Good courses: small class, large open-ended project
}

\begin{frame}
  \frametitle{Personal Statement}
  \begin{itemize}
    \item Convince admissions committee that you are a strong candidate
    \item Describe your research experience and interests
  \end{itemize}
\end{frame}
\note[itemize]
{%
  \item Other things: why you got into research, research plans, career plans
}

\begin{frame}
  \frametitle{Other Application Materials}
  \begin{itemize}
    \item Resume
    \item Grades
      \begin{itemize}
        \item May or may not be a cutoff
      \end{itemize}
    \item Graduate Record Examinations (GREs)
      \begin{itemize}
        \item Standardized test (similar to SATs)
        \item For US schools
      \end{itemize}
  \end{itemize}
\end{frame}
\note[itemize]
{%
  \item The most important thing is research experience
  \item Importance of grades or GREs depends on the school
  \item Grade cutoff at Waterloo is 78\% (CS and ECE)
  \item US schools will use GPA, so you'll have to convert
  \item Study for GREs early, before you get too specialized in CS and rusty in
        basic math
}

\section{Summary}
\begin{frame}
  \frametitle{\insertsection}
  \begin{itemize}
    \item What is graduate school?
      \begin{itemize}
        \item A job to do research: produce new knowledge
      \end{itemize}
    \item Reasons for graduate school?
      \begin{itemize}
        \item Make a name for yourself, fail in a safe environment, choose from
          more jobs
      \end{itemize}
    \item How to apply?
      \begin{itemize}
        \item Get research experience, which gets you essay material and letters
      \end{itemize}
  \end{itemize}
\end{frame}

\appendix
\begin{frame}
  \frametitle{Further Reading}
  \begin{itemize}
    \item \href{http://matt.might.net/articles/}{%
      \color{blue}{Matt Might's blog}
    }
    \item \href{http://pgbovine.net/phd.htm}{%
      \color{blue}{Philip Guo's articles}
    }
    \item \href{http://pgbovine.net/PhD-memoir.htm}{%
      \color{blue}{\textit{The Ph.D. Grind}}
    }
  \end{itemize}
\end{frame}

\end{document}
